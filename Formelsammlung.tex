\documentclass[a4paper]{article}

\usepackage[utf8]{inputenc}
%\usepackage[T1]{fontenc}  %westeuropäische Textcodierung
\usepackage{lmodern}  %schöneres Schriftbild
\usepackage[ngerman]{babel}  %Rechtschreibprüfung/Silbentrennung
\usepackage{graphicx}  %Verwendung von Bildern
%\usepackage{tabularx}  %Tabellenpaket
%\usepackage[all]{nowidow} %Hurenkinder vermeiden
\usepackage[style=ieee]{biblatex}  %erweiterte Bibtex-Version
\usepackage{csquotes}  %Hilfsbibliothek zu Biblatex
%\usepackage{amssymb}  %Sonderzeichen
%\usepackage{pxfonts} %griechisches Alphabet
%\usepackage{pdfpages} %Einbinden von PDFs
\usepackage{amsmath}
\usepackage{amssymb}
\usepackage{booktabs}
\usepackage{array}
\usepackage{makecell} %Zeilenumbrüche in Tabellen
\usepackage{multirow}
\usepackage{trfsigns} % Transformation Symbol o---o \laplace and \Laplace
\usepackage{esint}	%cyclic integrals

\usepackage{hyperref}  %Links einbinden (zuletzt aufrufen!)

\addbibresource{Literatur.bib}

\begin{document}


\title{\textbf{{\Huge Formelsammlung}}}
\author{Eike Osmers}
\maketitle
\thispagestyle{empty}
\newpage

\thispagestyle{empty}
\setcounter{page}{0}
\tableofcontents
\newpage
\setcounter{page}{1}

\section{Darstellungskonvention}

\begin{itemize}
\item[] Skalare Variablen in \textit{Kursivschrift}: $x, y, z$
\item[] Vektorielle Variablen mit einem Pfeil über der Variable: $\vec{a}$
\item[] Variablen für Matrizen in \textbf{fetter Schrift}: $\mathbf{A}$
\item[] Komplexe Variablen unterstrichen: 
\end{itemize}

\newpage
\section{Analysis}
\begin{table}[h]
\centering
%\resizebox{\textwidth}{!}{
\begin{tabular}{@{}>{\bfseries}lc@{}}
\toprule



\bottomrule
\end{tabular}
%}
\end{table}

\newpage
\section{Vektoranalysis}

\subsection{Vektoralgebra}

\begin{table}[h]
\centering
%\resizebox{\textwidth}{!}{
\begin{tabular}{@{}>{\bfseries}lc@{}}
\toprule

\makecell[l]{Skalarprodukt \\ {\normalfont {\tiny \textit{$\varphi$ ist der kleinere von $\vec{A}$ und}}} \\ {\normalfont {\tiny \textit{$\vec{B}$ eingeschlossene Winkel.}}}}
	& $\displaystyle\begin{aligned}
		&\vec{A}\cdot\vec{B} = ||\vec{A}|| \cdot ||\vec{B}|| \cdot \cos \varphi \\
		&{\scriptstyle\vec{A}\perp\vec{B}:\ \vec{A}\cdot\vec{B} = 0}
	\end{aligned}$ \\ \\
 
\makecell[l]{Kreuzprodukt \\ {\normalfont {\tiny \textit{$\varphi$ ist der kleinere von $\vec{A}$ und}}} \\ {\normalfont {\tiny \textit{$\vec{B}$ eingeschlossene Winkel.}}} \\ {\normalfont {\tiny \textit{$\vec{n}$ zeigt in Richtung der }}} \\ {\normalfont {\tiny \textit{Rechte-Hand-Regel.}}}}

	& $\displaystyle\begin{aligned}
		&\vec{A}\times\vec{B}= ||\vec{A}|| \cdot ||\vec{B}|| \cdot \sin \varphi \cdot \vec{n} \\
		&{\scriptstyle\vec{A}\parallel\vec{B}:\ \vec{A}\times\vec{B} = \vec{0}}
	\end{aligned}$ \\ \\
	
\makecell[l]{Richtungsvektor \\ {\normalfont {\tiny \textit{Zeigt von $\vec{A}$ auf $\vec{B}$.}}}}
	& $\displaystyle\vec{r}=\vec{B}-\vec{A}$ \\ \\
	
Tangentenvektor
	& $\displaystyle \frac{\partial \vec{x}}{\partial u}$ \\ \\

\makecell[l]{Fächennormal \\ {\normalfont {\tiny \textit{Steht immer senkrecht auf der Fläche.}}}}
	& $\displaystyle \vec{n} = \frac{\partial \vec{x}}{\partial u} \times \frac{\partial \vec{x}}{\partial v}$ \\

\bottomrule
\end{tabular}
%}
\end{table}

\newpage
\subsection{Koordinatensysteme}

\begin{figure}[h]
	\centering
	\includegraphics[width=13cm,keepaspectratio]{"Bilder/Koordinatensysteme"}
	\label{pic:Koordinatensysteme}
\end{figure}

\begin{table}[h]
\centering
\resizebox{\textwidth}{!}{
\begin{tabular}{@{}>{\bfseries}lccc@{}}
\toprule \\

  & \textbf{Kartesische Koordinaten} & \textbf{Zylinderkoordinaten} & \textbf{Kugelkoordinaten} \\ \\
  
\midrule   \\
                                        
Parametrisierung
	& $\displaystyle\begin{pmatrix}x \\ y \\ z \end{pmatrix}=\begin{pmatrix}x \\ y \\ z \end{pmatrix}$ 
	
	& $\displaystyle\begin{matrix} \begin{pmatrix}x \\ y \\ z \end{pmatrix}=\begin{pmatrix}\rho \cos (\varphi) \\ \rho \cos (\varphi) \\ z \end{pmatrix} \\ \\ \begin{pmatrix}\rho \\ \varphi \\ z \end{pmatrix}=\begin{pmatrix}\sqrt{x^2+y^2} \\ \operatorname{atan2}(\frac{y}{x}) \\ z \end{pmatrix} \end{matrix}$ 
	
	& $\displaystyle\begin{matrix} \begin{pmatrix}x \\ y \\ z \end{pmatrix}=\begin{pmatrix}r \cos (\varphi) \sin (\theta) \\ r \sin (\varphi) \sin (\theta) \\ r \sin (\theta) \end{pmatrix} \\ \\ \begin{pmatrix}r \\ \theta \\ \varphi \end{pmatrix}=\begin{pmatrix}\sqrt{x^2+y^2+z^2} \\ \arccos \frac{z}{\sqrt{x^2+y^2+z^2}} \\ \operatorname{atan2}\left(\frac{y}{x}\right) \end{pmatrix} \end{matrix}$ \\ \\

Definitionsbereich                         
	& $\displaystyle\begin{matrix} -\infty < x < \infty \\ -\infty < y < \infty \\ -\infty < z < \infty  \end{matrix} $ 
	& $\displaystyle\begin{matrix} 0 \leq \rho < \infty \\ 0 \leq \varphi \leq 2\pi \\ -\infty < z < \infty  \end{matrix}$ 
	& $\displaystyle\begin{matrix} 0 \leq r < \infty \\ 0 \leq \theta \leq \pi \\ 0 \leq \varphi \leq 2\pi \end{matrix} $ \\ \\

Transformationsmatrix                      
	& $\displaystyle\textbf{S}=\begin{bmatrix}1 & 0 & 0 \\ 0 & 1 & 0 \\ 0 & 0 & 1 \end{bmatrix} $ 
	& $\displaystyle\textbf{S}=\begin{bmatrix}\cos (\varphi) & -\sin (\varphi) & 0 \\ \sin (\varphi) & \cos (\varphi) & 0 \\ 0 & 0 & 1 \end{bmatrix} $ 
	& $\displaystyle\textbf{S}=\begin{bmatrix}\cos (\varphi) \sin (\theta) & \cos (\varphi) \cos (\theta) & -\sin (\varphi) \\ \sin (\varphi) \sin (\theta) & \sin (\varphi) \cos (\theta) & \cos (\varphi) \\ \cos (\theta) & -\sin (\theta) & 0 \end{bmatrix} $ \\ \\

inverse Transformationsmatrix              
	& $\displaystyle\textbf{S}^{\textbf{-1}}=\begin{bmatrix}1 & 0 & 0 \\ 0 & 1 & 0 \\ 0 & 0 & 1 \end{bmatrix} $ 
	& $\displaystyle\textbf{S}^{\textbf{-1}}=\begin{bmatrix}\cos (\varphi) & \sin (\varphi) & 0 \\ -\sin (\varphi) & \cos (\varphi) & 0 \\ 0 & 0 & 1 \end{bmatrix} $ 
	& $\displaystyle\textbf{S}^{\textbf{-1}}=\begin{bmatrix}\cos (\varphi) \sin (\theta) &  \sin (\varphi) \sin (\theta) & \cos (\theta) \\ \cos (\varphi) \cos (\theta) & \sin (\varphi) \cos (\theta) & -\sin (\theta) \\ -\sin (\varphi) & \cos (\varphi) & 0 \end{bmatrix} $ \\ \\

Transformation von Vektoren                
	& $\displaystyle\begin{pmatrix}a_x \\ a_y \\ a_z \end{pmatrix} = \begin{pmatrix}a_x \\ a_y \\ a_z \end{pmatrix}$ 
	& $\displaystyle\begin{matrix} \begin{pmatrix}a_x \\ a_y \\ a_z \end{pmatrix} = \textbf{S} \cdot \begin{pmatrix}a_\rho \\ a_\varphi \\ a_z \end{pmatrix} \\ \\ \begin{pmatrix}a_\rho \\ a_\varphi \\ a_z \end{pmatrix} = \textbf{S}^{\textbf{-1}} \cdot \begin{pmatrix}a_x \\ a_y \\ a_z \end{pmatrix} \end{matrix}$ 
	& $\displaystyle\begin{matrix} \begin{pmatrix}a_x \\ a_y \\ a_z \end{pmatrix} = \textbf{S} \cdot \begin{pmatrix}a_r \\ a_\theta \\ a_\varphi \end{pmatrix} \\ \\ \begin{pmatrix}a_r \\ a_\theta \\ a_\varphi \end{pmatrix} = \textbf{S}^{\textbf{-1}} \cdot \begin{pmatrix}a_x \\ a_y \\ a_z \end{pmatrix} \end{matrix}$ \\ \\

\midrule \\ \\

Einheitsvektoren in kart. Koordinaten      
	&  
	&  
	&  \\ \\

Bogenlängen-Element                        
	& $\displaystyle\mathrm{d}s^2 = \mathrm{d}x^2 + \mathrm{d}y^2 + \mathrm{d}z^2$ 
	& $\displaystyle\mathrm{d}s^2 = \mathrm{d}\rho^2 + \rho^2 \mathrm{d}\varphi^2 + \mathrm{d}z^2$ 
	& $\displaystyle\mathrm{d}s^2 = \mathrm{d}r^2 + r^2 \mathrm{d}\theta^2 + r^2 \sin^2(\theta) \mathrm{d}\varphi^2$ \\ \\

Linienelement entlang der Koordinatenlinie 
	&  
	&  
	&  \\ \\

Flächenelement der Koordinatenseitenfläche 
	&  
	&  
	&  \\ \\

Volumenelement                             
	& $\displaystyle\mathrm{d}V = \mathrm{d}x\ \mathrm{d}y\ \mathrm{d}z$ 
	& $\displaystyle\mathrm{d}V = \rho\, \mathrm{d}\rho\, \mathrm{d} \varphi\, \mathrm{d}z$ 
	& $\displaystyle\mathrm{d}V = r^2\sin^2(\theta)\, \mathrm{d}r\, \mathrm{d}\theta\,\mathrm{d}\varphi$ \\ \\
\bottomrule

\end{tabular}
}
\label{tab:Koordinatensysteme}
\end{table}

\newpage
\subsection{Differentialoperatoren} 

\cite{divgradcurl}

\begin{table}[h]
\resizebox{\textwidth}{!}{
\begin{tabular}{@{}>{\bfseries}lccc@{}}
\toprule \\

& \textbf{Kartesische Koordinaten} & \textbf{Zylinderkoordinaten} & \textbf{Kugelkoordinaten} \\ \\
  
\midrule \\

Nabla 
	& $\displaystyle \nabla = \begin{pmatrix} \frac{\partial}{\partial x} \\ \frac{\partial}{\partial y} \\ \frac{\partial}{\partial z}
\end{pmatrix} $
	&  $\displaystyle \nabla = \begin{pmatrix} \frac{\partial}{\partial \rho} \\ \frac{1}{\rho}\frac{\partial}{\partial \phi} \\ \frac{\partial}{\partial z}
\end{pmatrix} $
	&  $\displaystyle \nabla = \begin{pmatrix} \frac{\partial}{\partial r} \\ \frac{1}{r}\frac{\partial}{\partial \theta} \\ \frac{1}{r \sin \theta}\frac{\partial}{\partial \varphi}
\end{pmatrix} $\\ \\


\makecell[l]{Gradient \\ {\normalfont {\tiny \textit{(eines Skalarfeldes)}}}}
	& $\displaystyle\nabla a = \operatorname{grad}\ a = \begin{pmatrix}\frac{\partial a}{\partial x} \\ \frac{\partial a}{\partial y} \\ \frac{\partial a}{\partial z}\end{pmatrix}$ 
	& $\displaystyle\nabla a = \operatorname{grad}\ a = \begin{pmatrix}\frac{\partial a}{\partial \rho} \\ \frac{1}{\rho}\frac{\partial a}{\partial \theta} \\ \frac{\partial a}{\partial z}\end{pmatrix}$ 
	& $\displaystyle\nabla a = \operatorname{grad}\ a = \begin{pmatrix}\frac{\partial a}{\partial r} \\ \frac{1}{r}\frac{\partial a}{\partial \theta} \\ \frac{1}{r \sin (\theta)}\frac{\partial a}{\partial \varphi} \end{pmatrix}$ \\ \\


Divergenz $\displaystyle\mathbb{R}^3 \rightarrow \mathbb{R}$ 

	& $\displaystyle\begin{aligned}
		&\nabla \cdot \vec{a} =  \operatorname{div}\ \vec{a} = \\ 
		&\frac{\partial a_x}{\partial x} + \frac{\partial a_y}{\partial y} + \frac{\partial a_z}{\partial z}
		\end{aligned}$ 
	
	& $\displaystyle\begin{aligned}
		&\nabla \cdot \vec{a} =  \operatorname{div}\ \vec{a} = \\
		& \frac{1}{\rho}\frac{\partial}{\partial \rho} (\rho a_\rho) + \frac{1}{\rho} \frac{\partial a_\theta}{\partial \theta} + \frac{\partial a_z}{\partial z}
	\end{aligned}$ 
	
	& $\displaystyle\begin{aligned} 
		&\nabla \cdot \vec{a} = \operatorname{div}\ \vec{a} =   \\
		&\frac{1}{r^2}\frac{\partial}{\partial r} (r^2 a_r) + \frac{1}{r \sin (\theta)}\frac{\partial}{\partial \theta} \left(\sin (\theta\right) a_\theta) + \frac{1}{r \sin (\theta)}\frac{\partial a_\varphi}{\partial \varphi}
	\end{aligned}$ \\ \\


Rotation $\displaystyle\mathbb{R}^3 \rightarrow \mathbb{R}^3$ 

	& $\displaystyle\begin{aligned} 
		&\nabla \times \vec{a} = \operatorname{rot}\ \vec{a} = \\ 
		&\begin{pmatrix} \frac{\partial a_z}{\partial y} - \frac{\partial a_y}{\partial z} \\ \frac{\partial a_x}{\partial z} - \frac{\partial a_z}{\partial x} \\ \frac{\partial a_y}{\partial x} - \frac{\partial a_x}{\partial y} \end{pmatrix} 
	\end{aligned}$ 
	
	& $\displaystyle\begin{aligned} 
		&\nabla \times \vec{a} = \operatorname{rot}\ \vec{a} = \\
		&\begin{pmatrix} \frac{1}{\rho} \frac{\partial a_z}{\partial \theta} - \frac{\partial a_\theta}{\partial z} \\ \frac{\partial a_\rho}{\partial z} - \frac{\partial a_z}{\partial \rho} \\ \frac{1}{\rho} \frac{\partial}{\partial \rho} (\rho a_\theta) - \frac{1}{\rho} \frac{\partial a_\rho}{\partial \theta} \end{pmatrix}
	\end{aligned}$ 
	
	& $\displaystyle\begin{aligned} 
		&\nabla \times \vec{a} = \operatorname{rot}\ \vec{a} = \\ 
		&\frac{1}{r}\begin{pmatrix} \frac{1}{\sin (\varphi)} \left( \frac{\partial}{\partial \varphi} (\sin (\varphi) a_\theta) -  \frac{\partial a_\varphi}{\partial \theta} \right) \\ \frac{\partial}{\partial r} (r a_\varphi) -  \frac{\partial a_r}{\partial \varphi} \\ \frac{1}{\sin (\varphi)} \frac{\partial a_r}{\partial \theta} -  \frac{\partial}{\partial r} (r a_\theta) \end{pmatrix}
	\end{aligned}$ \\ \\

\bottomrule
\label{tab:divgradcurl}
\end{tabular}
}
\end{table}

\subsection{Integralsätze}

\begin{table}[h]
%\resizebox{\textwidth}{!}{
\begin{tabular}{@{}>{\bfseries}lccc@{}}
\toprule \\

\makecell[l]{Stokes \\ {\normalfont {\tiny \textit{$\partial A$ ist der Rand der Fläche $A$}}}}
	& $\displaystyle \oint_{\partial A} \vec{F}\, \mathrm{d}\vec{x} = \iint_A \nabla \times \vec{F}\, \mathrm{d}\vec{A}$ \\ \\
	
\makecell[l]{Gauß \\ {\normalfont {\tiny \textit{$\partial V$ ist die Oberfläche des Volumens $V$}}}}
	& $\displaystyle \oint_{\partial V} \vec{F}\, \mathrm{d}\vec{A} = \iiint_V \nabla \cdot \vec{F}\, \mathrm{d}V$ \\

\bottomrule
\label{tab:divgradcurl}
\end{tabular}
%}
\end{table}

\subsection{Was noch?}

(Wie löst man Kurven- und Oberflächenintegrale)\\

\section{Komplexe Funktionen}

\section{Lineare Algebra}

\subsection{Basiswechsel}

\newpage
\section{Signale und Systeme}

\subsection{Kontinuierliche Signale}

\begin{table}[h]
\centering
%\resizebox{\textwidth}{!}{
\begin{tabular}{@{}>{\bfseries}lc@{}}
\toprule

\makecell[l]{Energie eines Signals \\ {\normalfont {\tiny Energiesignal:}} \\ {\normalfont {\tiny \textit{endl. Energie, keine Leistung}}}}
	&$\displaystyle E_x = \int_{\vec{x}_1}^{\vec{x}_2} |x(t)|^2\,\mathrm{d}t$ \\ \\
	
\makecell[l]{(mittlere) Leistung eines Signals \\ {\normalfont {\tiny Leistungssignal:}} \\ {\normalfont {\tiny \textit{endl. Leistung, unendl. Energie}}}}
	& $\displaystyle P_x = \lim_{T\rightarrow \infty} \frac{1}{T} \int_{-\frac{T}{2}}^{\frac{T}{2}} |x(t)|^2\,\mathrm{d}t$ \\ \\
	
\midrule
	

\bottomrule
\end{tabular}
%}
\end{table}

\newpage
\subsubsection{Fourier-Transformation}

\begin{table}[h!]
\centering
\resizebox{\textwidth}{!}{
\begin{tabular}{@{}>{\bfseries}lc@{}}
\toprule

\makecell[l]{Relle Fourierreihe \\ {\normalfont {\tiny \textit{einer $T$-periodischen Funktion}}} \\ {\normalfont {\tiny i.d.R.: $T=2\pi$}}} 

	&$\begin{aligned}
		f(t) & = a_0 + \sum_{k=1}^\infty \left( a_k \cos \left( \frac{2\pi kt}{T} \right) + b_k \sin \left(\frac{2\pi kt}{T}\right) \right) \\
		a_0  & = \frac{1}{T} \int_0^{T} f(t)\,\mathrm{d}t \\
		a_k & =  \frac{2}{T} \int_0^{T} f(t) \cos \left( \frac{2\pi kt}{T} \right)\,\mathrm{d}t \\
		b_k & =  \frac{2}{T} \int_0^{T} f(t) \sin\left( \frac{2\pi kt}{T} \right)\,\mathrm{d}t
	\end{aligned}$ \\ \\
	
Symmetrieeigenschaften 
	& $\begin{aligned}
		a_k & =  0\Leftrightarrow f(t)\ \text{ungerade} \\
		b_k & =  0\Leftrightarrow f(t)\ \text{gerade}
	\end{aligned}$\\ \\
	
\makecell[l]{Komplexe Fourierreihe \\ {\normalfont {\tiny \textit{einer $T$-periodischen Funktion}}} \\ {\normalfont {\tiny i.d.R.: $T=2\pi$}}} 

	&$\begin{aligned}
		f(t) & = \sum_{k=1}^\infty \left( c_k\, \mathrm{e}^{\frac{\mathrm{j}2\pi kt}{T}} \right) \\
		c_k & = \frac{1}{T} \int_0^{T} f(t) \, \mathrm{e}^{-\frac{\mathrm{j}2\pi kt}{T}}\,\mathrm{d}t \\
	\end{aligned}$ \\ \\
	
\midrule \\

Fourier-Transformation
	& $\displaystyle F(\mathrm{j}\omega) =\mathcal{F}\{f(t\} = \int_{-\infty}^{\infty} f(t)\cdot \mathrm{e}^{-\mathrm{j}\omega t}\,\mathrm{d}t$ \\ \\
	
inverse Fourier-Transformation
	& $\displaystyle f(t) = \mathcal{F}^{-1}\{F(\mathrm{j}\omega)\} = \frac{1}{2\pi} \int_{-\infty}^\infty F(\mathrm{j}\omega) \cdot \mathrm{e}^{\mathrm{j}\omega t}\,\mathrm{d}\omega$ \\ \\ 
	
Eigenschaften
	&$\displaystyle \begin{aligned}
		f(t) & \text{ reell:}\  \Re\{F(\mathrm{j}\omega)\} \text{ gerade}, \Im\{F(\mathrm{j}\omega)\} \text{ ungerade} \\
		f(t) & \text{ gerade:}\  \Im\{F(\mathrm{j}\omega)\} = 0 \\
		f(t) & \text{ ungerade:}\  \Re\{F(\mathrm{j}\omega)\} = 0
	\end{aligned}$ \\ \\

\makecell[l]{Konvergenzbedingung \\ {\normalfont {\tiny \textit{$f(t)$ muss mind. quadratintegrierbar sein.}}}} 
	& $\displaystyle \int_{-\infty}^\infty |f(t)|^2\,\mathrm{d}t < \infty$ \\ \\
	
Ähnlichkeitssatz
	& $\displaystyle f(bt)\ \laplace\ \frac{1}{\vert b \vert} F\left(\frac{\mathrm{j}\omega}{b}\right)$\\ \\
	
\makecell[l]{Verschiebungssatz \\ {\normalfont {\tiny \textit{Zeitverschiebung im Zeitbereich}}} \\ {\normalfont {\tiny \textit{Phasenverschiebung im Frequenzbereich}}}}	
	& $\displaystyle f(t-t_0)\ \laplace\ \mathrm{e}^{-\mathrm{j}\omega t_0} F(\mathrm{j}\omega)$ \\ \\
	
\makecell[l]{Modulationssatz \\ {\normalfont {\tiny \textit{Multiplikation mit harm. Schwingung}}} \\ {\normalfont {\tiny \textit{Verschiebung im Frequenzbereich}}}}	
	& $\displaystyle \mathrm{e}^{\mathrm{j}\omega_0 t}f(t)\ \laplace\ F\left(\mathrm{j}(\omega - \omega_0)\right)$ \\ \\
	
Differentiationssatz
	& $\displaystyle \frac{d^n f(t)}{dt^n}\ \laplace\ (\mathrm{j}\omega)^n F(\mathrm{j}\omega)$ \\ \\

\bottomrule
\end{tabular}
}
\end{table}

\newpage
\subsubsection{Laplace-Transformation}

\begin{table}[h!]
\centering
%\resizebox{\textwidth}{!}{
\begin{tabular}{@{}>{\bfseries}lc@{}}
\toprule

\makecell[l]{komplexe Frequenz \\ {\normalfont {\tiny \textit{$\sigma$: Dämpfung/Verstärkung}}} \\ {\normalfont {\tiny \textit{$\omega$: Frequenz}}}}
	& $\displaystyle s = \sigma + \mathrm{j}\omega$ \\ \\
	
Einseitige Laplace-Transformation
	& $\displaystyle F(s) = \mathcal{L}\{f(t)\} = \int_{0}^\infty f(t)\, \mathrm{e}^{-st}\, \mathrm{d}t$ \\ \\
	
inverse Laplace-Transformation
	& $\displaystyle \begin{aligned}
		f(t) = \mathcal{L}^{-1} \{F(s)\} = \frac{1}{\mathrm{j}2\pi} \int_{c-\mathrm{j}\infty}^{c+\mathrm{j}\infty} F(s)\, \mathrm{e}^{st}\, \mathrm{d}s \\
		\\
		\text{{\small Alternativ: Rücktransformation}} \\ \text{{\small mittels Korrespondenz-Tabelle}} \\ \text{{\tiny ggf. Polydivision oder Partialbruchzerlegung notwendig}}
	\end{aligned}$ \\ \\
	
Konvergenzbedingung
	& $\displaystyle \int_{-\infty}^\infty |f(t)\, \mathrm{e}^{-st}|^2\,\mathrm{d}t < \infty$ \\ \\
	
\midrule \\
	
Verschiebungssatz
	& $\displaystyle f(t-t_0)\ \laplace\ \mathrm{e}^{-st_0} F(s)$ \\ \\
	
Dämpfungssatz
	& $\displaystyle\mathrm{e}^{bt} f(t)\ \laplace\ F(s-b)$ \\ \\
	
Integrationssatz
	& $\displaystyle\int_0^t f(\tau)\,\mathrm{d}\tau\ \laplace\ \frac{1}{s} F(s)$ \\ \\
	
Differentiationssatz
	& $\displaystyle\frac{\mathrm{d}^n f(t)}{\mathrm{d}t^n}\ \laplace\ s^n F(s) - \sum_{k=0}^{n-1} s^{n-k-1}\ \frac{\mathrm{d}^k f(0)}{\mathrm{d}t^k}$ \\ \\
	
Multiplikationssatz
	& $\displaystyle t^k f(t)\ \laplace\ (-1)^k\ \frac{\mathrm{d}^k F(s)}{\mathrm{d}s^k}$ \\ \\
	
Anfangswertsatz
	& $\displaystyle \lim_{t \rightarrow 0} f(t) = \lim_{s \rightarrow \infty} s F(s)$ \\ \\
	
Endwertsatz
	& $\displaystyle \lim_{t \rightarrow \infty} f(t) = \lim_{s \rightarrow 0} s F(s)$ \\ \\

\bottomrule
\end{tabular}
%}
\end{table}

\newpage
\subsection{Zeitdiskrete Signale}

\begin{table}[h]
\centering
%\resizebox{\textwidth}{!}{
\begin{tabular}{@{}>{\bfseries}lc@{}}
\toprule

Länge eines Signals
	& $\displaystyle N = N_2 - N_1 + 1 $ \\ \\
	
Faltungssumme
	& $\displaystyle \begin{aligned}
		 y[n] = & \sum_{k=-\infty}^\infty x[k] h[h-k] \\
		 = & \sum_{k=-\infty}^\infty x[n-k] h[k]
	\end{aligned}$ \\ \\

\makecell[l]{Zirkulare Zeitumkehr \\ {\normalfont {\tiny \textit{entlang eines Kreises mit Umfang $N-1$}}}}
 
	& $y[n] = x\left[ -n \operatorname{mod} N \right]  $ \\ \\
	
Zirkulare Verschiebung
	& $y[n] = x\left[ (n-n_0) \operatorname{mod} N \right] $ \\ \\
	
\midrule \\

\makecell[l]{Energie eines Signals \\ {\normalfont {\tiny Energiesignal:}} \\ {\normalfont {\tiny \textit{endl. Energie, keine Leistung}}}}
	& $\displaystyle E_x = \sum_{n=-\infty}^{\infty} |x[n]|^2$ \\ \\
	
\makecell[l]{mittlere Leistung \\ einer periodischen Folge \\ {\normalfont {\tiny Leistungssignal:}} \\ {\normalfont {\tiny \textit{endl. Leistung, unendl. Energie}}}}
	& $\displaystyle P_x = \frac{1}{N}\sum_{n=0}^{N-1} |x[n]|^2$ \\ \\
	
\makecell[l]{mittlere Leistung \\ einer aperiodischen Folge \\ {\normalfont {\tiny Leistungssignal:}} \\ {\normalfont {\tiny \textit{endl. Leistung, unendl. Energie}}}}
	& $\displaystyle P_x = \lim_{k \rightarrow \infty} \frac{1}{2k+1}\sum_{n=-k}^{k} |x[n]|^2$ \\ \\

\bottomrule
\end{tabular}
%}
\end{table}

\newpage
\subsubsection{z-Transformation}

Zur Angabe der z-Transformierten gehört immer die Angabe des Konvergenzgebietes (Region of Convergence, RoC), da es sonst keine eindeutige Umkehrung der z-Transformation gibt. \\

\begin{table}[h!]
\centering
\resizebox{\textwidth}{!}{
\begin{tabular}{@{}>{\bfseries}lcc@{}}
\toprule \\

 & \textbf{Transformation} & \textbf{Konvergenzgebiet} \\ \\
 
 \midrule \\

\makecell[l]{z-Transformation \\ {\normalfont {\tiny \textit{Geometr. Reihe hilft oft}}}}
	& $\displaystyle X(z) = \mathcal{Z} \{x[n]\}= \sum_{n=-\infty}^\infty x[n]\, z^{-n}$ 
	& $R_x$\\ \\

\makecell[l]{inverse z-Transformation \\ {\normalfont {\tiny \textit{$C$ muss um den Ursprung und}}} \\ {\normalfont {\tiny \textit{im Konvergenzgebiet von $X(z)$ liegen.}}}}
	& $\displaystyle x[n] = \mathcal{Z}^{-1} \{X(z)\} = \frac{1}{\mathrm{j}2\pi} \oint_C X(z)\ z^{n-1}\, \mathrm{d}z$ \\ \\
	
\midrule \\

Zeitverschiebung
	& $\displaystyle x[n-n_0]\ \laplace\ z^{-n_0}\, X(z)$ 
	& {\scriptsize $R_x, \text{evtl. Änderung bei} z=0 \text{ und } z=\infty$}\\ \\
	
Zeitumkehr
	& $\displaystyle x[-n] \laplace X\left(\frac{1}{z}\right)$ 
	& $\displaystyle \frac{1}{Rx}$\\ \\

Dämpfungssatz
	& $\displaystyle a^n\, x[n]\ \laplace\ X\left(\frac{z}{a}\right)$ 
	& $|a| R_x$\\ \\
	
Multiplikationssatz
	& $\displaystyle nx[n]\ \laplace\ -z\ \frac{\mathrm{d}X(z)}{\mathrm{d}z}$ 
	& {\scriptsize $R_x, \text{evtl. Änderung bei} z=0 \text{ und } z=\infty$} \\ \\
	
Anfangswertsatz
	& $\displaystyle \lim_{n \rightarrow 0} x[n]\ \laplace\ \lim_{z \rightarrow \infty} X(z)$ 
	&\\ \\
	
Endwertsatz
	& $\displaystyle \lim_{n \rightarrow \infty} x[n]\ \laplace\ \lim_{z \rightarrow 1} (z-1)X(z)$ 
	&\\ \\

\bottomrule
\end{tabular}
}
\end{table}

to-do: Stabilität der z-Transformation

\subsection{Zeit- \& Wertediskrete Signale}

\section{Elektrische Netzwerke}

Modified Node Analysis (PaBe Kochrezept im Ordner)

\subsection{passive Bauelemente}

Ersatzschaltbilder von R und C und L \\

0. Ordnung R: R \\

1. Ordnung R: L+R oder C||R \\

2. Ordnung R: (R||C)+L oder (L+R)||C \\

1. Ordnung C: C+R oder C||R \\

Universal ESB C: Ls + Rs + (Rp || R(f) || C) \\

1. Ordnung L: L + R \\

Güte Reihen- und Parallelschwingkreis \\

\subsection{Netzwerkparameter}

siehe PaBe Kapitel 5

\newpage
\section{Klassische Elektrodynamik}

\cite{Georg.Felder}
\cite{Marinescu}

\begin{table}[h]
\resizebox{\textwidth}{!}{
\begin{tabular}{@{}>{\bfseries}lccc@{}}

\toprule

 & \textbf{Skalare Größe} &\multicolumn{2}{c}{\textbf{Vektorielle Größe}}  \\
 
 & & \textbf{Differentielle Form} & \textbf{Integrale Form} \\

\midrule

\multicolumn{4}{c}{\textbf{Maxwell Gleichungen}} \\



\midrule

\multicolumn{4}{c}{\textbf{Elektrostatik}} \\

\makecell[l]{Coulomb'sches Kraftgesetz \\ {\normalfont {\tiny Kraftvektoren zweier Ladungen}} \\ {\normalfont {\tiny zeigen von einander weg.}} \\ {\normalfont {\tiny $n>2$: Superposition}}}
	& $\displaystyle F_C =  \frac{1}{4\pi \varepsilon_0}\cdot\frac{q_1 q_2}{r^2}$
	&
	 & $\displaystyle\vec{F}_{C_{12}} = \frac{q_1 q_2}{4 \pi \varepsilon_0}\cdot\frac{\vec{x}_2 - \vec{x}_1}{||\vec{x}_2 - \vec{x}_1||^3}$\\ \\
	 
\makecell[l]{Elektrische Feldstärke \\ $\displaystyle\left[\vec{E}\right] = \frac{V}{m}$}
	& 
	& $\displaystyle \vec{E}(\vec{x}) =  -\operatorname{grad}(\varphi_e)$
	& $\displaystyle \vec{E}(\vec{x}) =  \frac{\vec{F}_{C_{12}}}{q_2} = \frac{q_1}{4 \pi \varepsilon_0}\cdot\frac{\vec{x} - \vec{x}_1}{||\vec{x} - \vec{x}_1||^3}$\\ \\
	
\makecell[l]{Elektrische Potential(feld) \\ $\displaystyle\left[\varphi_e\right] = V$} \\{\normalfont {\tiny Gleiches Potential auf Äquipotentialflächen}}
	& $\displaystyle \varphi_e(\vec{x}) = \frac{q_1}{4 \pi \varepsilon_0}\cdot\frac{1}{||\vec{x} - \vec{x}_1||}$
	& 
	&\\ \\
	
\makecell[l]{Elektrische Spannung \\ $\displaystyle[U] = V$}
	& $\displaystyle\begin{aligned}
		U_{12} = & \varphi_e(\vec{x}_2) - \varphi_e(\vec{x}_1) \\
		= & - \int_{\vec{x}_1}^{\vec{x}_2} \vec{E}\, d\vec{s}
	\end{aligned}$
	& 
	&\\ \\
	 
\midrule \\ \\
	
\makecell[l]{Linienladungsdichte \\ $[\lambda] = \frac{C}{m}$}
	& 
	&$\displaystyle\lambda =  \frac{dq_e}{ds}$ 
	& $\displaystyle q_e =  \int \lambda\,ds$ \\ \\
	
\makecell[l]{Flächenladungsdichte \\ $[\sigma] = \frac{C}{m^2}$}
	& 
	&$\displaystyle\sigma =  \frac{dq_e}{dA}$ 
	&$\displaystyle q_e =  \iint \sigma\,dA$ \\ \\
	
\makecell[l]{Volumenladungsdichte \\ $[\rho] = \frac{C}{m^3}$}
	& 
	& $\displaystyle\rho =  \frac{dq_e}{dV}$ 
	& $\displaystyle q_e =  \iiint \rho\,dV$ \\ \\

\bottomrule

Maxwellgleichungen für verschiedene Anwendungsfälle (allgemein, Elektrostatik, quasistatisch, stationär & quasistationär (siehe PaBe Kap. 1), 

\label{tab:Elektrostatik}
\end{tabular}
}
\end{table}

\newpage
\section{Elektronik}



\subsection{Operationsverstärker}

\section{Hochfrequenztechnik}

\section{Fehlerrechnung}

\newpage
\printbibliography

\end{document}